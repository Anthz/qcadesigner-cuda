\documentclass[a4paper,12pt]{article}

\usepackage{graphicx}
\usepackage{latexsym}
\usepackage{amssymb}
\usepackage{amsmath}
\usepackage{color}
\usepackage[english]{babel}
\usepackage[latin1]{inputenc}
\usepackage{fancyhdr}
%%Algorithm
\usepackage[longend,linesnumbered,boxruled,algosection,lined]{settings/algorithm2e}
\usepackage{settings/firstpage} % modificare questo file per i diplomi e se si ha
                        % un solo correlatore

%% \hyphenation{} is used to force the 
\hyphenation{}

\newtheorem{Definition}{Definition}[section]

\pagestyle{fancy}

\begin{document}

\title{Standard MicroLab: \LaTeX }
\providecommand{\annoacc}{2010}

% genera la prima pagina
%\titlfp

%\maketitle
%\titlepage

\newpage

\tableofcontents

\newpage

\listoffigures

%\newpage

%\listoftables

\headsep 2cm
\hoffset -1cm
\textwidth 15cm
\setlength{\headwidth}{\textwidth}
  %uncomment the following line to remove chapter in the header
  %\fancyhead[R]{}

   \fancyhead[L]{
   %\includegraphics*[width=1in]{img/scritta}
   %with this you can move up or down the title in order not to overlap with chapter title.
   \raisebox{5mm}{	
      \begin{tabular}{l}
          %Prova di titolo
       \end{tabular}
   }
}



%============= ABSTRACT ====================
\newpage
\begin{abstract}
2 righe sul lavoro, ->VENDITI BENE
\end{abstract}


%============================= INTRODUCTION =================================
\newpage
\chapter{Project Introduction}\label{sec:intro}




%============================= SECTIONS =================================

\newpage
\section{Qca Overview}\label{sec:i}
Quantum Dot Cellular Automata (QCA) are a quantum extension to the classical notion of Cellular Automata. 

In computability theory, we identify a Cellular Automata (CA) as a Turing-complete abstract machine model consisting of a grid of cells each of which can be in any of a finite number of states. For every cell in the grid, moreover, we define its neighborhood, which is the set of "near enough" cells. The evolution of this automaton is defined by the transition of the state of the cells, which is a function of both the status of the cell and the status of its neighborhood.

What makes a QCA different from a classical CA is that its physical implementation is regulated by the quantum laws of physics instead of the classical ones. Let's take a closer look to a QCA. As you can see in \figurename~\ref{fig:qca} a QCA is made of three main components: electron wells, crystalline substrate and electrons themselves. Here we depict the charge of the electrons on the wells/dots by blurred blue clouds instead of the more common (and classical) representation of small, crisply defined points. This is accurate because the Quantum Mechanics says that at this near atomic scale, charge (on electrons and protons) does not come in sharp little balls, but instead is correctly representable only in a probabilistic way via the wave function (Schr�dinger equation).

\begin{figure}[h!bt]
	\centerline{\includegraphics[width=\textwidth]{img/qca.png}}
	\caption{A single Quantum Dot Cellular Automata (QCA) cell. In green the crystalline (SiGe - Silicon Germanium) substrate, where are clearly visible, in squared position, the four wells meant for accomodating that many electrons. Even though different configurations of the dots are possible, this is by far the most commonly studied structure. The blue cloud represents the amplitude of probability of finding an electron in that position.}
	\label{fig:qca}
\end{figure}

Charges of the same type (positive or negative) still repel one another so our two clouds of charge will move as far from one another as possible by running to opposite corners of the QCA. As you can see in Figure \ref{fig:qcaevo}, vs tunnel effect vs clock phase) ...... (tunnel effect, quantum mechanics)

\begin{figure}[h!bt]
	\centerline{\includegraphics[width=8cm]{img/qcaevo.png}}
	\caption{The evolution of the state of a QCA cell. From left to right, from top to bottom: the lower of the two wires (on the left in every image) is polarized, inducing (second image) the lower left electron to flow to the righest side of the QCA (and precisely inside the corresponding well). This, in turn, induces the upper right electron to move farthest from it, inside the upper left well (third and fourth image).}
	\label{fig:qcaevo}
\end{figure}

Apart from the possibility of using QCA for building quantum computing, which is something quite not at hand yet, QCAs can be seen as a powerful alternatives to silicon-based devices to represent data. Depending on the fabrication technique employed, in fact, QCAs are theorethically able to switch at an order of magnitude around the Thz, to be packaged at very high density levels and to disspate very small amounts of power. This, along with the existance of demo systems already proven working, justifies their current and future development as an attractive way of improving nowadays computing performance bottlenecks.

One of the universities interested and involved into the development of this technology, among the others, is the University of Columbia, and in particular its Microsystems and Nanotechnology Group (MiNa Group). The foundation of our work is their implementation of a simulator (both logical and physical) of QCA based cicuits and is the subject of the next section.


\newpage
\section{Cuda Overview}\label{sec:i}


\newpage
\chapter{implementation}\label{sec:implementation}
\section{First approach}
The original source code of QCADesigner was downloaded from Mina website (ref). We attempted to make it compile as it was but we did not manage to solve several compilation errors. So we started to focus on the identification of the core algorithm supporting the tool in order to obtain a working batch simulator executable on CPU. This operation took us some weeks of work. Meanwhile we were able to deeply analyze the code. We made some hypothesis on the location of possible bottlenecks, we identified the data structures used to represent circuits and started to consider possible transformations that had to be done in order to obtain fast accessible and light weight data structures allocable on the GPU global memory.

\section{CPU algorithm and profiling}

\newpage
\chapter{Results}\label{sec:results}
%correctness - tests done on a set of circuits
%baseline: the original cpu code execution time
%improvements: the mean and the peak speedup
%graphs: speedup vs input size + ESTRAPOLAZIONE
%        speedup vs cell size + ESTRAPOLAZIONE
%speedup prediction accuracy as #cells*2^i/#threads / (real ex. time)
In order to evaluate the performance of CudaQCAD, several tests were executed. These tests aimed to show strengths and fallacies of the simulator both in comparison with the original QCAD and in the usage of the resources offered by the architecture. In the next sections will be described the Hardware used for the tests and the organization of the tests themselves.

\section{Hardware Description}
All the tests were performend using the "Lucifero" Workstation, located in the "MicroLab" of Politecnico di Milano. \\
\begin{description}
\item[CPU] Intel Xeon E5345
\item[GPU] Nvidia Testa C1060
\end{description} 

\section{Test 1: CPU vs GPU}
The most meaningful test executed is certainly the comparison between the original QCAD and CudaQCAD. Both the version of the simulator were tested on a pool of QCAs composed by several real circuits of different complexity. This test was designed in order to evaluate:

\begin{description}
\item[Correctness] of the two simulators.
\item[Coherency] of the results.
\item[Performance] in terms of execution time.
\end{description}    

Considering the first two points, the test has shown that both the simulators obtain the same results for all the circuits of the pool and that these results were alligned with the well known behaviour of the circuits.\\
Regarding to the performance, it was noticed a considerable speed-up - 30x of peak - of the GPU version over CPU version only when the number of cells composing the simulated circuits was greater than 7000. For circuits with a lower number of cells, the original QCAD outperform CudaQCAD, Fig x.
 
\begin{figure}[h!bt]
	\centerline{\includegraphics[width=0.7\textwidth]{img/xeonvstesla.png}}
	\caption{Performance comparison between Intel Xeon and Nvidia Tesla.}
	\label{fig:NvidiaGPUsLogicalOrg}
\end{figure}

\begin{figure}[h!bt]
        \centerline{\includegraphics[width=0.7\textwidth]{img/speedup.png}}
        \caption{Overall SpeedUp.}
        \label{fig:NvidiaGPUsLogicalOrg}
\end{figure}

This result was not surprising. Until the number of cells in the circuits is lower than a given threshold - 7000 with the used architecture - the overhead due to the device setup time is bigger than the gain due to the parallelism. However, considering that the original goal was to speed-up the simulation of those circuits that requires an amount of time too high to be considered acceptable for pratical purpouses, this test can be considered passed.

\section{Test 2: Memory Transfers}
As stated in chapter X, memory transfers between host and device are the most important factor, and often the most limiting, in order to achieve good performance in Cuda.\\
A well built GPU algorithm should not waste more than 20\% of the whole execution time for Host-Device I/O. Considering this assumption, the results achieved by CudaQCAD are actually good: all the tested circuits showed a memory transfer rate lower than 10\%, even those under the threshold of 7000 cells (Fig X.X, X.Y).

\begin{figure}[h!bt]
        \centerline{\includegraphics[width=\textwidth]{img/GPUTimeSummaryPlotMUX42.png}}
        \caption{Memory transfers for MUX42 circuit (21551 cells).}
        \label{fig:OccupancyAnalysis}
\end{figure}


\begin{figure}[h!bt]
        \centerline{\includegraphics[width=\textwidth]{img/GPUTimeSummaryPlotNAND.png}}
        \caption{Memory Tranfer for NAND circuit (1642 cells).}
        \label{fig:OccupancyAnalysis}
\end{figure}
   
\section{Test 3: GPU Occupancy}
This test aims to prove the effective usage of the GPU resourcers. The occupancy is the ratio of active warps to the maximum number of warps supported on a multiprocessor of the GPU. As stated in chapter 3, each multiprocessor on the device has a set of N registers available for use by Cuda thread programs. These registers are a shared resource that are allocated among the thread blocks executing on a multiprocessor. The Cuda compiler attempts to minimize register usage to maximize the number of thread blocks that can be active in the machine simultaneously. If a program tries to launch a kernel for which the registers used per thread are too many, the number of threads in concurrent execution will decrease, as consequentely the performance will do.\\
Unfortunalely there is no way to automatic tune the execution parameters in order to achieve the full occupancy. For this reason, several attempts were performed and profiled, Fig X. Considering that the occupancy can be fairy evaluated only when the amout of computations is big enought to require the usages of all the available resources, for this test a circuit with a number of cells greater than the threshold was choosen.\\ 
The first attempt was unsatifactory, due to an occupancy of only 50\%. For this reason the block size was progressively enlarged till achieving the maximum occupancy supported. Unfortunately, even with a larger block size, the occupancy could not be moved to more than 75\%, due to the register pressure.\\
In order to further improve the occupacy, the kernel function was re-designed with special attention to the number of registers per threads, as described in chapter X. The new version of the kernel lead to much better results and the occupation was bring up to 100\%.
    
\begin{figure}[h!bt]
        \centerline{\includegraphics[width=0.7\textwidth]{img/OccupancyAnalysis.png}}
        \caption{Occupancy Analysis. Data extracted from Nvidia Occupancy Calculator.}
        \label{fig:OccupancyAnalysis}
\end{figure}

\section{Test 4: Divergent Branches}
The number of divergent branches can seriously impact on the performance of the simulator. As a rule of thumb the percentage of divergent branches must be lower than 15\% to be considered acceptable. In all the performed test, CudaQCAD was actually under this threshold, with an average divergent branches rate of 9\%.   



\newpage
\chapter{Conclusions}\label{sec:conclusions}
FUTURE WORK: -> SIAMO IN CONTATTO CON UN DOTTORANDO DEL MINA DISPOSTO A DARCI UNA MANO A RISOLVERE IL PROBLEMA DELL'OSCILLAZIONE

\section{You can do as many section as you want}
You can insert a definition
	\begin{Definition}
		DRESD: Dynamic Reconfigurability in Embedded System Design
	\end{Definition}
	
\newpage

\section{Another section}
	\subsection{And a subsection}
		In this sub section I will include many images just to make the list of figures meaningful.

			\begin{figure}[h!tb]
				\centerline {\includegraphics[width=9cm,height=4.5cm]{img/scritta.png}}
				\caption{Caption of this DRESD image.}
				\label{fig:leet}
			\end{figure}

\newpage

\section{You can do as many section as you want}
	\subsection{And also many sub}
		\subsubsection{Or again more nested}
			\begin{itemize}
				\item item 1
				\item item 2
				\item item 3
			\end{itemize}			

\newpage
			
\section{You can do as many section as you want}
This is an example of table. You can see it in Table~\ref{tab:TableName}. 

 \begin{table}[h!tb]
   \centering \caption{Table caption}
   \label{tab:TableName}
   \vskip 0.2cm
   %%
   \scalebox{0.90}{
	    %% The {|c|c|c|c|c|} define the number of columns.
	    %% c means centered
	    %% | defines a vertical line between two columns 
	    \begin{tabular}{|c|c|c|}
	      \hline
	      Col 1 & Col 2 & Col2  \\
	      %% \\ force a newline without creating an horizontal line
	      Dim C.1 & Dim C.3 & Dim C.3 \\
	      %% \hline create an horizontal line between two rows
	      \hline
	      Data 1.1 & Data 1.2 & Data 1.3 \\
	      \hline
	      Data 2.1 & Data 2.2 & Data 2.3 \\
	      \hline
	      Data 3.1 & Data 3.2 & Data 3.3 \\  
	      \hline
	    \end{tabular}
	 }
 \end{table}

\newpage

\section{You can do as many section as you want}
The following equation has been created using \verb.$$.: $ y_1 = a*x + b$.

%% \smallskip introduces a small vertical skip
\smallskip

%% \noindent is used to eliminate the blankline and the paragraph settings 
\noindent
This is a more complex equation: \[ y_2 = a*x + b \], created using \verb.\[ \].

%% \bigskip introduces a big vertical skip
\bigskip

\noindent
This is, (\ref{eq:eeq1}), an enumerated equation: 

	\begin{equation}\label{eq:eeq1}
		y_3 = a*x + b
	\end{equation}

, created using:

	\begin{verbatim}
		\begin{equation}\label{eq:eeq1}
		    y_3 = a*x + b
		\end{equation}
	\end{verbatim}

%============================= BIBLIOGRAPHY =================================

\newpage

%this is the standard for bibliography references for IEEE Transactions	
\bibliographystyle{settings/IEEEtran}
\bibliography{DocBibliography}

\end{document}
